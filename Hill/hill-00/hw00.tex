\documentclass[a4paper]{article}
\usepackage[letterpaper, margin=1in]{geometry} % page format
\usepackage{listings} % this package is for including code
\usepackage{graphicx} % this package is for including figures
\usepackage{amsmath}  % this package is for math and matrices
\usepackage{amsfonts} % this package is for math fonts
\usepackage{tikz} % for drawings
\usepackage{hyperref} % for urls

\title{Homework 00}
\author{Charles Hill}
\date{9/5/16}

\begin{document}
\lstset{language=Python}

\maketitle

\section{Problem 1}
To find the the value for x that maximizes g(x) we need to take the derivative and solve for 0. In the case that value of x is 4.
\begin{equation}
\begin{align}
  g(x) = -3x^{2} + 24x - 30 \\
	g'(x) = -6x +24 \\
	0 = -6x + 24 \\
	-24 = -6x \\
	x = 4 \\
\end{align}	
\end{equation}

\section{Problem 2}
We need to differentiate the equation twice. 
\begin{equation}
\begin{align}
  f(x) = 3x_{0}^{3} + 2x_{0}x_{1}^{2} - 8 \\
	f'(x_{0}) = 6x_{0}^{2} + 2x_{1}^{2} \\
	f'(x_{1}) = 4x_{0}x_{1}\\
\end{align}	
\end{equation}

\section{Problem 3}
(a) We can not multiple the two matrices as they are both 2x3 and therefore do not line up when they are multiplied. \\
(b) We can use code to multiply the transpose of A with B. \\

\begin{lstlisting}[frame=single]
import numpy as np

a = [[1, 4], [2, -1], [-3, 3]]
b = [[-2, 0, 5], [0, -1, 4]]

print np.dot(a, b)
\end{lstlisting}

\begin{lstlisting}
[[-2 -4 21]
[-4 1 6]]
[6 -3 3]
\end{lstlisting}

The resulting matrix has a rank of 2.

\section{Problem 4}
A simple Gaussian is the function that creates the form of the bell curve that is commonly used in statistics. \\
A multivariate Guassian allows for the distribution to be multidimensional. \\
Bernoulli is the discrete distribution that has only two values. Generally it is representated as a 0 and a 1 where 1 could be called "`success"' \\
The binomial distribution occurs when obtaining 'n' successes out 'N' Bernoulli trials. \\
Exponential distribution is the probability distribution that describes the time between events in a process that occurs continuously and independently at a constant rate. \\

\section{Problem 6}
If X~N(2,3). The expected value would be 2.5 because it is a normal distribution. \\

\section{Problem 7}
(a) If y = 1.1, that the expected x* when Z = N (N being the natural numbers) would be 1. \\
(b) Problem7Picture.jpg is the solution. The red dot is the where x* would be.

\section{Problem 8}
(a) Using $e^{-y}$ as y goes to infinity, the value will only become a smaller and smaller fraction. As y goes to negative infinity then all the values will be 0. Therefore the value for the integral from negative infinity to infinity will just be 1. \\
(b) The value of y would be e because as y approaches infinity y would be multiplied by $e^{-y}$ and thus the value would just be e. As y approaches negative infinity the value would just be 0. \\
(c) The variance would be 2.95 \\
(d) The expected value would just be e. \\

\end{document}
